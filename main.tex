\documentclass{anstrans}
%%%%%%%%%%%%%%%%%%%%%%%%%%%%%%%%%%%
\title{Replicating and Evaluating A Deep Larning Approach to Nuclear Fuel Transmutation in a Fuel Cycle Simulator}
\author{Samar E. Abdelkawy, Kathryn D. Huff}

\institute{
Advanced Reactors and Fuel Cycles Group, University of Illinois,
Urbana, IL, selsafy2@illinois.edu}


% Optional disclaimer: remove this command to hide
% \disclaimer{Notice: this manuscript is a work of fiction. Any resemblance to
% actual articles, living or dead, is purely coincidental.}

%%%% packages and definitions (optional)
\usepackage{graphicx} % allows inclusion of graphics
\usepackage{booktabs} % nice rules (thick lines) for tables
\usepackage{microtype} % improves typography for PDF
\usepackage[acronym,toc]{glossaries}
\newacronym[longplural={metric tons of heavy metal}]{MTHM}{MTHM}{metric ton of heavy metal}
\newacronym{ABM}{ABM}{agent-based modeling}
\newacronym{ACDIS}{ACDIS}{Program in Arms Control \& Domestic and International Security}
\newacronym{AHTR}{AHTR}{Advanced High Temperature Reactor}
\newacronym{ANDRA}{ANDRA}{Agence Nationale pour la gestion des D\'echets RAdioactifs, the French National Agency for Radioactive Waste Management}
\newacronym{APP}{APP}{Abbott Power Plant}
\newacronym{ANL}{ANL}{Argonne National Laboratory}
\newacronym{API}{API}{application programming interface}
\newacronym{ARCH}{ARCH}{autoregressive conditional heteroskedastic}
\newacronym{ARE}{ARE}{Aircraft Reactor Experiment}
\newacronym{ARFC}{ARFC}{Advanced Reactors and Fuel Cycles}
\newacronym{ARMA}{ARMA}{autoregressive moving average}
\newacronym{ASME}{ASME}{American Society of Mechanical Engineers}
\newacronym{ATWS}{ATWS}{Anticipated Transient Without Scram}
\newacronym{BDBE}{BDBE}{Beyond Design Basis Event}
\newacronym{BIDS}{BIDS}{Berkeley Institute for Data Science}
\newacronym{BOL}{BOL}{Beginning-of-Life}
\newacronym{BSD}{BSD}{Berkeley Software Distribution}
\newacronym{CAFCA}{CAFCA}{ Code for Advanced Fuel Cycles Assessment }
\newacronym{CASL}{CASL}{Consortium for Advanced Simulation of Light Water Reactors}
\newacronym{CDTN}{CDTN}{Centro de Desenvolvimento da Tecnologia Nuclear}
\newacronym{CEA}{CEA}{Commissariat \`a l'\'Energie Atomique et aux \'Energies Alternatives}
\newacronym{CI}{CI}{continuous integration}
\newacronym{CNEC}{CNEC}{Consortium for Nonproliferation Enabling Capabilities}
\newacronym{CNEN}{CNEN}{Comiss\~{a}o Nacional de Energia Nuclear}
\newacronym{CNERG}{CNERG}{Computational Nuclear Engineering Research Group}
\newacronym{COSI}{COSI}{Commelini-Sicard}
\newacronym{COTS}{COTS}{commercial, off-the-shelf}
\newacronym{CSNF}{CSNF}{commercial spent nuclear fuel}
\newacronym{CTAH}{CTAHs}{Coiled Tube Air Heaters}
\newacronym{CUBIT}{CUBIT}{CUBIT Geometry and Mesh Generation Toolkit}
\newacronym{CURIE}{CURIE}{Centralized Used Fuel Resource for Information Exchange}
\newacronym{DAG}{DAG}{directed acyclic graph}
\newacronym{DANESS}{DANESS}{Dynamic Analysis of Nuclear Energy System Strategies}
\newacronym{DBE}{DBE}{Design Basis Event}
\newacronym{DESAE}{DESAE}{Dynamic Analysis of Nuclear Energy Systems Strategies}
\newacronym{DHS}{DHS}{Department of Homeland Security}
\newacronym{DOE}{DOE}{Department of Energy}
\newacronym{DRACS}{DRACS}{Direct Reactor Auxiliary Cooling System}
\newacronym{DRE}{DRE}{dynamic resource exchange}
\newacronym{DSNF}{DSNF}{DOE spent nuclear fuel}
\newacronym{DYMOND}{DYMOND}{Dynamic Model of Nuclear Development }
\newacronym{EBS}{EBS}{Engineered Barrier System}
\newacronym{EDZ}{EDZ}{Excavation Disturbed Zone}
\newacronym{EIA}{EIA}{U.S. Energy Information Administration}
\newacronym{EPA}{EPA}{Environmental Protection Agency}
\newacronym{EP}{EP}{Engineering Physics}
\newacronym{FCO}{FCO}{Fuel Cycle Options}
\newacronym{FCT}{FCT}{Fuel Cycle Technology}
\newacronym{FCWMD}{FCWMD}{Fuel Cycle and Waste Management Division}
\newacronym{FEHM}{FEHM}{Finite Element Heat and Mass Transfer}
\newacronym{FEPs}{FEPs}{Features, Events, and Processes}
\newacronym{FHR}{FHR}{Fluoride-Salt-Cooled High-Temperature Reactor}
\newacronym{FLiBe}{FLiBe}{Fluoride-Lithium-Beryllium}
\newacronym{GCAM}{GCAM}{Global Change Assessment Model}
\newacronym{GDSE}{GDSE}{Generic Disposal System Environment}
\newacronym{GDSM}{GDSM}{Generic Disposal System Model}
\newacronym{GENIUSv1}{GENIUSv1}{Global Evaluation of Nuclear Infrastructure Utilization Scenarios, Version 1}
\newacronym{GENIUSv2}{GENIUSv2}{Global Evaluation of Nuclear Infrastructure Utilization Scenarios, Version 2}
\newacronym{GENIUS}{GENIUS}{Global Evaluation of Nuclear Infrastructure Utilization Scenarios}
\newacronym{GPAM}{GPAM}{Generic Performance Assessment Model}
\newacronym{GRSAC}{GRSAC}{Graphite Reactor Severe Accident Code}
\newacronym{GUI}{GUI}{graphical user interface}
\newacronym{HLW}{HLW}{high level waste}
\newacronym{HPC}{HPC}{high-performance computing}
\newacronym{HTC}{HTC}{high-throughput computing}
\newacronym{HTGR}{HTGR}{High Temperature Gas-Cooled Reactor}
\newacronym{IAEA}{IAEA}{International Atomic Energy Agency}
\newacronym{IEMA}{IEMA}{Illinois Emergency Mangament Agency}
\newacronym{INL}{INL}{Idaho National Laboratory}
\newacronym{IPRR1}{IRP-R1}{Instituto de Pesquisas Radioativas Reator 1}
\newacronym{IRP}{IRP}{Integrated Research Project}
\newacronym{ISFSI}{ISFSI}{Independent Spent Fuel Storage Installation}
\newacronym{ISRG}{ISRG}{Independent Student Research Group}
\newacronym{JFNK}{JFNK}{Jacobian-Free Newton Krylov}
\newacronym{LANL}{LANL}{Los Alamos National Laboratory}
\newacronym{LBNL}{LBNL}{Lawrence Berkeley National Laboratory}
\newacronym{LCOE}{LCOE}{levelized cost of electricity}
\newacronym{LDRD}{LDRD}{laboratory directed research and development}
\newacronym{LFR}{LFR}{Lead-Cooled Fast Reactor}
\newacronym{LGPL}{LGPL}{Lesser GNU Public License}
\newacronym{LLNL}{LLNL}{Lawrence Livermore National Laboratory}
\newacronym{LMFBR}{LMFBR}{Liquid-Metal-cooled Fast Breeder Reactor}
\newacronym{LOFC}{LOFC}{Loss of Forced Cooling}
\newacronym{LOHS}{LOHS}{Loss of Heat Sink}
\newacronym{LOLA}{LOLA}{Loss of Large Area}
\newacronym{LP}{LP}{linear program}
\newacronym{LWR}{LWR}{Light Water Reactor}
\newacronym{MARKAL}{MARKAL}{MARKet and ALlocation}
\newacronym{MA}{MA}{minor actinide}
\newacronym{MCNP}{MCNP}{Monte Carlo N-Particle code}
\newacronym{MILP}{MILP}{mixed-integer linear program}
\newacronym{MIT}{MIT}{the Massachusetts Institute of Technology}
\newacronym{MOAB}{MOAB}{Mesh-Oriented datABase}
\newacronym{MOOSE}{MOOSE}{Multiphysics Object-Oriented Simulation Environment}
\newacronym{MOX}{MOX}{mixed oxide}
\newacronym{MSBR}{MSBR}{Molten Salt Breeder Reactor}
\newacronym{MSRE}{MSRE}{Molten Salt Reactor Experiment}
\newacronym{MSR}{MSR}{Molten Salt Reactor}
\newacronym{NAGRA}{NAGRA}{National Cooperative for the Disposal of Radioactive Waste}
\newacronym{NCSA}{NCSA}{National Center for Supercomputing Applications}
\newacronym{NEAMS}{NEAMS}{Nuclear Engineering Advanced Modeling and Simulation}
\newacronym{NEUP}{NEUP}{Nuclear Energy University Programs}
\newacronym{NFCSim}{NFCSim}{Nuclear Fuel Cycle Simulator}
\newacronym{NFC}{NFC}{Nuclear Fuel Cycle}
\newacronym{NGNP}{NGNP}{Next Generation Nuclear Plant}
\newacronym{NMWPC}{NMWPC}{Nuclear MW Per Capita}
\newacronym{NNSA}{NNSA}{National Nuclear Security Administration}
\newacronym{NPRE}{NPRE}{Department of Nuclear, Plasma, and Radiological Engineering}
\newacronym{NQA1}{NQA-1}{Nuclear Quality Assurance - 1}
\newacronym{NRC}{NRC}{Nuclear Regulatory Commission}
\newacronym{NSF}{NSF}{National Science Foundation}
\newacronym{NSSC}{NSSC}{Nuclear Science and Security Consortium}
\newacronym{NUWASTE}{NUWASTE}{Nuclear Waste Assessment System for Technical Evaluation}
\newacronym{NWF}{NWF}{Nuclear Waste Fund}
\newacronym{NWTRB}{NWTRB}{Nuclear Waste Technical Review Board}
\newacronym{OCRWM}{OCRWM}{Office of Civilian Radioactive Waste Management}
\newacronym{ORION}{ORION}{ORION}
\newacronym{ORNL}{ORNL}{Oak Ridge National Laboratory}
\newacronym{PARCS}{PARCS}{Purdue Advanced Reactor Core Simulator}
\newacronym{PBAHTR}{PB-AHTR}{Pebble Bed Advanced High Temperature Reactor}
\newacronym{PBFHR}{PB-FHR}{Pebble-Bed Fluoride-Salt-Cooled High-Temperature Reactor}
\newacronym{PEI}{PEI}{Peak Environmental Impact}
\newacronym{PH}{PRONGHORN}{PRONGHORN}
\newacronym{PI}{PI}{Principal Investigator}
\newacronym{PNNL}{PNNL}{Pacific Northwest National Laboratory}
\newacronym{PRIS}{PRIS}{Power Reactor Information System}
\newacronym{PRKE}{PRKE}{Point Reactor Kinetics Equations}
\newacronym{PSPG}{PSPG}{Pressure-Stabilizing/Petrov-Galerkin}
\newacronym{PWAR}{PWAR}{Pratt and Whitney Aircraft Reactor}
\newacronym{PWR}{PWR}{Pressurized Water Reactor}
\newacronym{PyNE}{PyNE}{Python toolkit for Nuclear Engineering}
\newacronym{PyRK}{PyRK}{Python for Reactor Kinetics}
\newacronym{QA}{QA}{quality assurance}
\newacronym{RDD}{RD\&D}{Research Development and Demonstration}
\newacronym{RD}{R\&D}{Research and Development}
\newacronym{RELAP}{RELAP}{Reactor Excursion and Leak Analysis Program}
\newacronym{RIA}{RIA}{Reactivity Insertion Accident}
\newacronym{RIF}{RIF}{Region-Institution-Facility}
\newacronym{SAM}{SAM}{Simulation and Modeling}
\newacronym{SCF}{SCF}{Software Carpentry Foundation}
\newacronym{SFR}{SFR}{Sodium-Cooled Fast Reactor}
\newacronym{SINDAG}{SINDA{\textbackslash}G}{Systems Improved Numerical Differencing Analyzer $\backslash$ Gaski}
\newacronym{SKB}{SKB}{Svensk K\"{a}rnbr\"{a}nslehantering AB}
\newacronym{SNF}{SNF}{spent nuclear fuel}
\newacronym{SNL}{SNL}{Sandia National Laboratory}
\newacronym{SNM}{SNM}{Special Nuclear Material}
\newacronym{STC}{STC}{specific temperature change}
\newacronym{SUPG}{SUPG}{Streamline-Upwind/Petrov-Galerkin}
\newacronym{SWF}{SWF}{Separations and Waste Forms}
\newacronym{SWU}{SWU}{Separative Work Unit}
\newacronym{SandO}{S\&O}{Signatures and Observables}
\newacronym{THW}{THW}{The Hacker Within}
\newacronym{TRIGA}{TRIGA}{Training Research Isotope General Atomic}
\newacronym{TRISO}{TRISO}{Tristructural Isotropic}
\newacronym{TSM}{TSM}{Total System Model}
\newacronym{TSPA}{TSPA}{Total System Performance Assessment for the Yucca Mountain License Application}
\newacronym{UDB}{UDB}{Unified Database}
\newacronym{UFD}{UFD}{Used Fuel Disposition}
\newacronym{UML}{UML}{Unified Modeling Language}
\newacronym{UNFSTANDARDS}{UNFST\&DARDS}{Used Nuclear Fuel Storage, Transportation \& Disposal Analysis Resource and Data System}
\newacronym{UOX}{UOX}{uranium oxide}
\newacronym{UQ}{UQ}{uncertainty quantification}
\newacronym{US}{US}{United States}
\newacronym{UW}{UW}{University of Wisconsin}
\newacronym{VISION}{VISION}{the Verifiable Fuel Cycle Simulation Model}
\newacronym{VV}{V\&V}{verification and validation}
\newacronym{WIPP}{WIPP}{Waste Isolation Pilot Plant}
\newacronym{YMG}{YMG}{Young Members Group}
\newacronym{YMR}{YMR}{Yucca Mountain Repository Site}
\newacronym{NEI}{NEI}{Nuclear Energy Institute}
%\newacronym{<++>}{<++>}{<++>}
%\newacronym{<++>}{<++>}{<++>}


\usepackage{algorithm}
\usepackage{algpseudocode}
\usepackage{array}

\usepackage{xspace}
\usepackage{multirow}

\newcommand{\cycamore}{\textsc{Cycamore}\xspace}
\newcommand{\cyclus}{\textsc{Cyclus}\xspace}
\newcommand{\SN}{S$_N$}
\renewcommand{\vec}[1]{\bm{#1}} %vector is bold italic
\newcommand{\vd}{\bm{\cdot}} % slightly bold vector dot
\newcommand{\grad}{\vec{\nabla}} % gradient
\newcommand{\ud}{\mathop{}\!\mathrm{d}} % upright derivative symbol

\begin{document}
%%%%%%%%%%%%%%%%%%%%%%%%%%%%%%%%%%%%%%%%%%%%%%%%%%%%%%%%%%%%%%%%%%%%%%%%%%%%%%%%
\section{Introduction}
The nuclear fuel cycle is fundamental to nuclear energy production and reactor operations. Understanding the fuel cycle is essential for assessing how different enrichments and burnups affect the isotopic composition of spent nuclear fuel (SNF), including its concentration of various isotopes and decay heat. These factors play a crucial role in determining appropriate strategies for SNF management, including storage, reprocessing, or disposal. Optimizing these processes requires accurate depletion modeling to inform decision-making on waste handling and potential resource utilization\cite{yacout_modeling_2005}.

This work replicates the study conducted by Bae et al. (2020)\cite{bae_deep_2020}, where the authors trained a neural network model on the Unified Database (UDB)\cite{peterson_used_2013} for Pressurized Water Reactor (PWR) Mixed Oxide (MOX) fuel with varying enrichments and burnups. Their results demonstrated that the model provided a balance between fidelity and computational efficiency compared to other nuclear fuel cycle simulators. However, since the UDB has since been updated, there is an opportunity to validate their approach using new data. Bae et al. noted that, ideally, their model should be tested against a dataset it was not trained on, but this was not feasible at the time, leading them to validate the model using a subset of their training data. In this study, we implement their trained neural network model within the CYCLUS framework and evaluate its performance using the updated dataset. This allows us to assess whether the model still outperforms traditional recipe-based methods in fuel cycle simulations.





%%%%%%%%%%%%%%%%%%%%%%%%%%%%%%%%%%%%%%%%%%%%%%%%%%%%%%%%%%%%%%%%%%%%%%%%%%%%%%%%
\section{\cyclus}

CYCLUS \cite{huff_fundamental_2016} is an open-source, agent-based nuclear fuel cycle (NFC) simulation framework designed for flexibility and extensibility. Unlike traditional simulators that rely on fixed system models, CYCLUS treats each facility—such as reactors, enrichment plants, and storage sites—as independent agents that interact dynamically. These agents operate under predefined rules, exchanging materials through a market-based mechanism called the Dynamic Resource Exchange \cite{gidden_methodology_2016}.

A key strength of CYCLUS is its modular architecture, which allows users to define and deploy custom facility models, known as archetypes, implemented in C++ or Python. Standard NFC processes, such as enrichment, reprocessing, and storage, are available in the Cycamore\cite{carlsen_cycamore_2014} repository, while additional community-developed archetypes extend its capabilities for specialized applications, including spent fuel transmutation and diversion modeling. This modularity enables CYCLUS to simulate a wide range of NFC scenarios, making it a powerful tool for analyzing the impact of policy decisions, technology changes, and resource availability on nuclear energy systems.
%%%%%%%%%%%%%%%%%%%%%%%%%%%%%%%%%%%%%%%%%%%%%%%%%%%%%%%%%%%%%%%%%%%%%%%%%%%%%%%%%%
\section{Unified Database}
\section{Acknowledgments}
% \begin{figure}[!htbp]
%   \centering
%   \includegraphics[width=0.4\textwidth]{../images/cyclus_logo.png}
%   \label{fig:cyclus_logo}
% \end{figure}


% \begin{algorithm}
% \caption{Greedy Reactor Deployment Algorithm}
% \begin{algorithmic}[1]
%     \State Initialize demand
%     \While{demand exists}
%         \State Select the largest reactor that does not exceed demand
%         \State Deploy reactors until the next reactor exceeds demand
%         \State Update demand
%     \EndWhile
% \end{algorithmic}
% \end{algorithm}

% \begin{algorithm}
%   \caption{Random Reactor Deployment Algorithm}
%   \begin{algorithmic}[1]
%       \State Initialize demand
%       \While{demand exists}
%           \State Randomly deploy a reactor that does not exceed demand
%           \State Update demand
%       \EndWhile
%   \end{algorithmic}
%   \end{algorithm}

%   \begin{algorithm}
%     \caption{Random + Greedy Reactor Deployment Algorithm}
%     \begin{algorithmic}[1]
%         \State Initialize demand
%         \While{demand exists}
%             \State Randomly deploy a reactor
%             \If{demand is exceeded}
%                 \State Remove last reactor
%                 \If{demand still exists}
%                     \State Select the largest reactor that does not exceed demand
%                     \State Deploy until the next reactor exceeds demand
%                     \State Update demand
%                 \EndIf
%             \EndIf
%         \EndWhile
%     \end{algorithmic}
%     \end{algorithm}

% \begin{subequations} \label{eqs:fullTransport}
% \begin{multline} \label{eq:fullTransportVol}
%   \vec{\Omega}\vd \grad \psi(\vec{x}, \vec{\Omega})
%   + \sigma(\vec{x}) \psi (\vec{x}, \vec{\Omega})
% \\ =
%   \frac{\sigma_s(\vec{x})}{4\pi} \int_{4\pi} \psi(\vec{x},\vec{\Omega}')
%   \ud\Omega' + \frac{q(\vec{x})}{4\pi}
%   \equiv \frac{1}{4\pi} Q(\vec{x}) \,,
% \end{multline}
% inside $\vec{x} \in V$, $\vec{\Omega} \in 4\pi$, with an incident boundary
% condition
% \begin{equation} \label{eq:fullTransportBndy}
%   \psi(\vec{x}, \vec{\Omega}) = \psi^b(\vec{x}, \vec{\Omega}) \,,
%  \quad \vec{x} \in \partial V, \ \vec{\Omega} \vd \vec{n} < 0\,.
% \end{equation}
% \end{subequations}

%%%%%%%%%%%%%%%%%%%%%%%%%%%%%%%%%%%%%%%%%%%%%%%%%%%%%%%%%%%%%%%%%%%%%%%%%%%%%%%%
% \section{Results and Analysis}
% Table \ref{tab:enrichment_levels} shows the various levels of enrichment for uranium that we will use in this work.

% \begin{table}[!htbp]
%    \centering
%    \caption{Enrichment levels and their ranges.}
%    \label{tab:enrichment_levels}
%    \begin{tabular}{c c}
%       \hline
%       \textbf{Enrichment Level} & \textbf{Range [\%  $^{235}$U]} \\
%       \hline
%       Natural & < 0.711 \\
%       \gls{leu} & 0.711-5 \\
%       \gls{leup} & 5-10 \\
%       \gls{haleu} & 10-20 \\
%       % \gls{heu} & $\geq$ 20  \\
%       \hline
%    \end{tabular}
% \end{table}

%%%%%%%%%%%%%%%%%%%%%%%%%%%%%%%%%%%%%%%%%%%%%%%%%%%%%%%%%%%%%%%%%%%%%%%%%%%%%%%%
% \subsection{Subsection Goes Here (Heading B)}
% The user must manually capitalize initial letters of a subsection heading.

% For those who like equations in their papers, \LaTeX\ is a good choice. Here is
% an equation for the Marshak diffusion boundary condition:
% \begin{equation} \label{eq:marshak}
%   4 J^- = \phi + 2 D \vec{n} \vd \grad \phi \,.
% \end{equation}
% If we so choose, we can effortlessly reference the equation later.

% Another paragraph starts with Eq.~\eqref{eq:marshak} and sets $J^-$ to zero, a
% vacuum boundary condition:
% \begin{equation*}
%   0 = \phi + \frac{2}{3} \frac{1}{\sigma} \vec{n} \vd \grad \phi \,.
% \end{equation*}
% The extrapolation distance is $2/3$. A more detailed asymptotic analysis yields
% an extrapolation distance of about $0.71045$.



% Later on, we can include a table, even one that spans two columns such as
% Table~\ref{tab:widetable}.
%%%%%%%%%%%%%%%%%%%%%%%%%%%%%%%%%%%%%%%%
% \begin{table*}[htb]
%   \centering
%   \caption{Example of a Really Wide Table that Might Not Normally Fit in the Document}
%   \begin{tabular}{llllllllll}\toprule
%       & $\phi_T(0)$      & $\phi_T(10)$      & $\phi_T(20)$      &
%       $\phi_D(0)$      & $\phi_D(10)$      & $\phi_D(20)$      & $\rho$      &
%       $\varepsilon$      & $N_\text{it}$
% \\ \midrule
% $c=0.999$  & 0.9038 & 20.63 & 31.24 & 0.9087 & 20.63 & 31.23 & 0.2192 & $10^{-7}$ & 15
% \\
% $c=0.990$  & 0.3675 & 13.04 & 24.7 & 0.3696 & 13.04 & 24.69 & 0.2184 & $10^{-7}$ & 15
% \\
% $c=0.900$  & 0.009909 & 4.776 & 17.64 & 0.009984 & 4.786 & 17.63 & 0.2118 & $10^{-7}$ & 14
% \\
% $c=0.500$  & $6.069\times 10^{-5}$ & 2.212 & 15.53 & 6.213$\times 10^{-5}$ & 2.239 & 15.53 & 0.2068 & $10^{-7}$ & 13
% \\
% \bottomrule
% \end{tabular}
%   \label{tab:widetable}
% \end{table*}
%%%%%%%%%%%%%%%%%%%%%%%%%%%%%%%%%%%%%%%%
% Notice how the table reference uses a Roman numeral
% for its numbering scheme, whereas the figure reference uses an Arabic numeral.
% For one-column tables, use the \verb|table| environment; two-column tables use
% \verb|table*|. The same applies to figures.

%%%%%%%%%%%%%%%%%%%%%%%%%%%%%%%%%%%%%%%%%%%%%%%%%%%%%%%%%%%%%%%%%%%%%%%%%%%%%%%%
% \section{Conclusions (Heading A)}

% The included ANS style file and this clear example file are a panacea for
% the hours of headache that invariably results from formatting a document in
% Microsoft Word.

%%%%%%%%%%%%%%%%%%%%%%%%%%%%%%%%%%%%%%%%%%%%%%%%%%%%%%%%%%%%%%%%%%%%%%%%%%%%%%%%
\appendix
% \section{Appendix}

% Numbering in the appendix is different:
% \begin{equation} \label{eq:appendix}
%   2 + 2 = 5\,.
% \end{equation}
% and another equation:
% \begin{equation} \label{eq:appendix2}
%   a + b = c\,.
% \end{equation}

%%%%%%%%%%%%%%%%%%%%%%%%%%%%%%%%%%%%%%%%%%%%%%%%%%%%%%%%%%%%%%%%%%%%%%%%%%%%%%%%
%\section{Acknowledgments}

%%%%%%%%%%%%%%%%%%%%%%%%%%%%%%%%%%%%%%%%%%%%%%%%%%%%%%%%%%%%%%%%%%%%%%%%%%%%%%%%
\bibliographystyle{ans}
\bibliography{bibliography}
\end{document}